\section{Antecedentes}
\label{section:antecedentes}

Craig W. Reynolds, publica “Steering Behaviors For Autonomous Characters” en este trabajo se explica la forma de crear un modelo computacional para el movimiento 
coordinado de grupos de caracteres autónomos (llamados boids).
 
Estas entidades son capaces de moverse por su mundo de una manera improvisada y realista. Para ello el modelo tiene que respetar ciertas reglas que aseguren las condiciones reales 
del movimiento de caracteres coordinados. Siendo las dos reglas más importantes las siguientes:
No puede existir una inteligencia superior al resto.
Cada entidad se mueve de modo independiente según las reglas del modelo y su ponderación respectiva.
 
Gracias a estas reglas, y modificando los valores ponderables de los comportamientos la simulación se asemejara mucho a la conducta real de diferentes especies, desde bancos de 
peces a bandadas de aves. Existiendo siempre una cohesión para que se desplacen en manada, una alineación para que se dirijan en la misma dirección y una separación para evitar que
choquen los unos con los otros.
 
Además, aunando diferentes comportamientos se pueden crear otros mas complejos que pueden dar lugar a situaciones donde las entidades sean capaces de evitar obstáculos del mundo o 
seguir un camino prefijado como una manada.
 
Algunas de las características más importantes del modelo son: la falta de predictibilidad en un lapso de tiempo considerable y la formación de comportamientos emergentes, es decir, 
que no estan planeados y que pueden sorprender al propio creador.


