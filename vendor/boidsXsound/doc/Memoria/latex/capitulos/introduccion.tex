\section{Introducción}
\label{chap:introduccion}

En el presente proyecto, se busca crear un entorno virtual de pruebas para poder analizar los comportamientos asociados a estados de ánimo en un conjunto de caracteres autónomos. Dichos comportamientos dependerán del nivel de estrés y este, a su vez, cambiará en función de lo que perciban a través de sus sentidos.
Para dotar a la simulación de un mayor realismo se crearan fuentes de audio en tiempo real capaces de analizar y reproducir el audio cargado en ellas.  La frecuencia resultante del análisis será lo que llegue a los caracteres autónomos mediante su sistema auditivo, permitiendo el estudio de los comportamientos en función de lo que perciban.\\  

Los objetivos que se han marcado son:
\begin{itemize}
 \item Recrear el sistema de vida artificial presentado por Craig Reynolds en su trabajo “Steering Behaviors For Autonomous Characters”
 \item Ser capaces de cargar varias fuentes de audio, para posteriormente ser analizadas y después reproducidas. 
 \item Calcular la frecuencia fundamental de una señal de audio.
 \item Crear un sistema auditivo y de comunicación para caracteres autónomos. 
 \item Diseñar una interface adecuada para poder realizar el mayor número de pruebas.  
 \item Analizar el comportamiento de los caracteres en función de unos parámetros definidos por el usuario a través de la interface.
\end{itemize}

La memoria consta de cuatro capítulos dedicados cada uno a un aspecto del proyecto. En el primero se comenta de forma detallada los sistemas existentes más conocidos en el área de la aplicación. 

En el siguiente, se detalla el proceso de análisis llevado a cabo para la realizacion del proyecto y el cumplimiento de los objetivos marcados al inicio del mismo. 

En el capítulo cuatro se describen en profundidad la metodología de trabajo, la tecnología utilizada ,las clases principales del programa, el funcionamiento detallado de la aplicación y una descripción de la interface.
                   
Finalmente, las conclusiones acerca de los resultados obtenidos en este trabajo y las futuras ampliaciones para poder continuar la labor detallada en la introducción.

              
